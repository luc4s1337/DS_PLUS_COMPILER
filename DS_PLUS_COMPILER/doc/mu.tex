\documentclass{article}

%encoding
%--------------------------------------
\usepackage[T1]{fontenc}
\usepackage[utf8]{inputenc}
%--------------------------------------

%Portuguese-specific commands
%--------------------------------------
\usepackage[portuguese]{babel}
%--------------------------------------

\usepackage{graphicx}
\usepackage{indentfirst}
\usepackage{hyperref}
\usepackage{tabularx}


\title{Estudo Dirigido Compiladores}

\author{Lucas Ferreira de Almeida}

\date{13/10/2021}

\begin{document}
	
	\maketitle
	
	\tableofcontents
	
	\pagebreak
	
	\section{Tokens}
	
	\subsection{Identificadores}
	
	\begin{tabularx}{0.8\textwidth} { 
			| >{\raggedright\arraybackslash}X 
			| >{\centering\arraybackslash}X 
			| >{\raggedleft\arraybackslash}X | }
		\hline
		Identificador & ID  \\
		\hline
	\end{tabularx}
	
	\subsection{Literais}
	
	\begin{tabularx}{0.8\textwidth} { 
			| >{\raggedright\arraybackslash}X 
			| >{\centering\arraybackslash}X 
			| >{\raggedleft\arraybackslash}X | }
		\hline
		Inteiros & \texttt{LIT\_INT}  \\
		\hline
		Reais & \texttt{LIT\_FLT}   \\
		\hline
		Caractere & \texttt{LIT\_CHAR}  \\
		\hline
		String & \texttt{LIT\_STR}   \\
		\hline
		Boolean & \texttt{LIT\_BOOL}  \\
		\hline
	\end{tabularx}
	
	\subsection{Palavras Reservadas}
	
	\begin{tabularx}{0.8\textwidth} { 
			| >{\raggedright\arraybackslash}X 
			| >{\centering\arraybackslash}X 
			| >{\raggedleft\arraybackslash}X | }
		\hline
		void & \texttt{PR\_VOID}  \\
		\hline
		int & \texttt{PR\_INT}  \\
		\hline
		float & \texttt{PR\_FLT}  \\
		\hline
		char & \texttt{PR\_CHAR}  \\
		\hline
		bool & \texttt{PR\_BOOL}  \\
		\hline
		if & \texttt{PR\_IF}  \\
		\hline
		then & \texttt{PR\_THEN}  \\
		\hline
		else & \texttt{PR\_ELSE}  \\
		\hline
		end-if & \texttt{PR\_ENDIF}  \\
		\hline
		for & \texttt{PR\_FOR}  \\
		\hline
		while & \texttt{PR\_WHILE}  \\
		\hline
		do & \texttt{PR\_DO}  \\
		\hline
		loop & \texttt{PR\_LOOP}  \\
		\hline
		return & \texttt{PR\_RETURN}  \\
		\hline
		break & \texttt{PR\_BREAK}  \\
		\hline
		continue & \texttt{PR\_CONTINUE}  \\
		\hline
		goto & \texttt{PR\_GOTO}  \\
		\hline
		true & \texttt{PR\_TRUE}  \\
		\hline
		false & \texttt{PR\_FALSE}  \\
		\hline
		var & \texttt{PR\_VAR}  \\
		\hline
		main & \texttt{PR\_MAIN}  \\
		\hline
		scan & \texttt{PR\_SCAN}  \\
		\hline
		scanln & \texttt{PR\_SCANLN}  \\
		\hline
		print & \texttt{PR\_PRINT}  \\
		\hline
		println & \texttt{PR\_PRINTLN}  \\
		\hline
	\end{tabularx}
	
	\subsection{Pontuação}
	
	\begin{tabularx}{0.8\textwidth} { 
			| >{\raggedright\arraybackslash}X 
			| >{\centering\arraybackslash}X 
			| >{\raggedleft\arraybackslash}X | }
		\hline
		, & \texttt{VIRGULA} \\
		\hline
		; & \texttt{PONTO\_VIRGULA} \\
		\hline
		( & \texttt{ABRE\_PARENTESES} \\
		\hline
		) & \texttt{FECHA\_PARENTESES} \\
		\hline
		[ & \texttt{ABRE\_COLCHETES} \\
		\hline
		] & \texttt{FECHA\_COLCHETES} \\
		\hline
		{ & \texttt{ABRE\_CHAVES} \\
			\hline
		} & \texttt{FECHA\_CHAVES} \\
		\hline
	\end{tabularx} 
	
	\subsection{Operadores}
	
	\begin{tabularx}{0.8\textwidth} { 
			| >{\raggedright\arraybackslash}X 
			| >{\centering\arraybackslash}X 
			| >{\raggedleft\arraybackslash}X | }
		\hline
		+ & \texttt{OP\_SOMA} \\
		\hline
		- & \texttt{OP\_SUB} \\
		\hline
		* & \texttt{OP\_MULT} \\
		\hline
		/ & \texttt{OP\_DIV} \\
		\hline
		\% & \texttt{OP\_MOD} \\
		\hline
		? & \texttt{OP\_TER} \\
		\hline
		! & \texttt{OP\_NEG} \\
		\hline
		. & \texttt{OP\_PONTO} \\
		\hline
		< & \texttt{OP\_MENOR} \\
		\hline
		> & \texttt{OP\_MAIOR} \\
		\hline
		== & \texttt{OP\_IGUAL} \\
		\hline
		!= & \texttt{OP\_DIF} \\
		\hline
		<= & \texttt{OP\_MEN\_IGUAL} \\
		\hline
		>= & \texttt{OP\_MAI\_IGUAL} \\
		\hline
		= & \texttt{OP\_ATRI} \\
		\hline
		+= &  \texttt{OP\_ADIC\_IGUAL} \\
		\hline
		-= &  \texttt{OP\_SUB\_IGUAL} \\
		\hline
		++ &  \texttt{OP\_INC} \\
		\hline
		-- &  \texttt{OP\_DEC} \\
		\hline
		\&\& &  \texttt{OP\_AND} \\
		\hline
		|| &  \texttt{OP\_OR} \\
		\hline
	\end{tabularx}
	
	\section{Diagramas de transição}
	
	\subsection{Identificadores}
	\subsection{Literais}
	\subsection{Pontuação}
	\subsection{Palavras Reservadas}
	\subsection{Operadores}
	
\end{document}